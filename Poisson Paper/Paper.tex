% Options for packages loaded elsewhere
\PassOptionsToPackage{unicode}{hyperref}
\PassOptionsToPackage{hyphens}{url}
\PassOptionsToPackage{dvipsnames,svgnames,x11names}{xcolor}
%
\documentclass[
  12pt]{article}

\usepackage{amsmath,amssymb}
\usepackage{iftex}
\ifPDFTeX
  \usepackage[T1]{fontenc}
  \usepackage[utf8]{inputenc}
  \usepackage{textcomp} % provide euro and other symbols
\else % if luatex or xetex
  \usepackage{unicode-math}
  \defaultfontfeatures{Scale=MatchLowercase}
  \defaultfontfeatures[\rmfamily]{Ligatures=TeX,Scale=1}
\fi
\usepackage{lmodern}
\ifPDFTeX\else  
    % xetex/luatex font selection
\fi
% Use upquote if available, for straight quotes in verbatim environments
\IfFileExists{upquote.sty}{\usepackage{upquote}}{}
\IfFileExists{microtype.sty}{% use microtype if available
  \usepackage[]{microtype}
  \UseMicrotypeSet[protrusion]{basicmath} % disable protrusion for tt fonts
}{}
\usepackage{xcolor}
\setlength{\emergencystretch}{3em} % prevent overfull lines
\setcounter{secnumdepth}{5}
% Make \paragraph and \subparagraph free-standing
\ifx\paragraph\undefined\else
  \let\oldparagraph\paragraph
  \renewcommand{\paragraph}[1]{\oldparagraph{#1}\mbox{}}
\fi
\ifx\subparagraph\undefined\else
  \let\oldsubparagraph\subparagraph
  \renewcommand{\subparagraph}[1]{\oldsubparagraph{#1}\mbox{}}
\fi


\providecommand{\tightlist}{%
  \setlength{\itemsep}{0pt}\setlength{\parskip}{0pt}}\usepackage{longtable,booktabs,array}
\usepackage{calc} % for calculating minipage widths
% Correct order of tables after \paragraph or \subparagraph
\usepackage{etoolbox}
\makeatletter
\patchcmd\longtable{\par}{\if@noskipsec\mbox{}\fi\par}{}{}
\makeatother
% Allow footnotes in longtable head/foot
\IfFileExists{footnotehyper.sty}{\usepackage{footnotehyper}}{\usepackage{footnote}}
\makesavenoteenv{longtable}
\usepackage{graphicx}
\makeatletter
\def\maxwidth{\ifdim\Gin@nat@width>\linewidth\linewidth\else\Gin@nat@width\fi}
\def\maxheight{\ifdim\Gin@nat@height>\textheight\textheight\else\Gin@nat@height\fi}
\makeatother
% Scale images if necessary, so that they will not overflow the page
% margins by default, and it is still possible to overwrite the defaults
% using explicit options in \includegraphics[width, height, ...]{}
\setkeys{Gin}{width=\maxwidth,height=\maxheight,keepaspectratio}
% Set default figure placement to htbp
\makeatletter
\def\fps@figure{htbp}
\makeatother

\addtolength{\oddsidemargin}{-.5in}%
\addtolength{\evensidemargin}{-1in}%
\addtolength{\textwidth}{1in}%
\addtolength{\textheight}{1.7in}%
\addtolength{\topmargin}{-1in}%
\makeatletter
\@ifpackageloaded{caption}{}{\usepackage{caption}}
\AtBeginDocument{%
\ifdefined\contentsname
  \renewcommand*\contentsname{Table of contents}
\else
  \newcommand\contentsname{Table of contents}
\fi
\ifdefined\listfigurename
  \renewcommand*\listfigurename{List of Figures}
\else
  \newcommand\listfigurename{List of Figures}
\fi
\ifdefined\listtablename
  \renewcommand*\listtablename{List of Tables}
\else
  \newcommand\listtablename{List of Tables}
\fi
\ifdefined\figurename
  \renewcommand*\figurename{Figure}
\else
  \newcommand\figurename{Figure}
\fi
\ifdefined\tablename
  \renewcommand*\tablename{Table}
\else
  \newcommand\tablename{Table}
\fi
}
\@ifpackageloaded{float}{}{\usepackage{float}}
\floatstyle{ruled}
\@ifundefined{c@chapter}{\newfloat{codelisting}{h}{lop}}{\newfloat{codelisting}{h}{lop}[chapter]}
\floatname{codelisting}{Listing}
\newcommand*\listoflistings{\listof{codelisting}{List of Listings}}
\makeatother
\makeatletter
\makeatother
\makeatletter
\@ifpackageloaded{caption}{}{\usepackage{caption}}
\@ifpackageloaded{subcaption}{}{\usepackage{subcaption}}
\makeatother
\ifLuaTeX
  \usepackage{selnolig}  % disable illegal ligatures
\fi
\usepackage[]{natbib}
\bibliographystyle{agsm}
\usepackage{bookmark}

\IfFileExists{xurl.sty}{\usepackage{xurl}}{} % add URL line breaks if available
\urlstyle{same} % disable monospaced font for URLs
\hypersetup{
  pdftitle={Integrated Likelihood Inference in Poisson Distributions},
  pdfauthor={Timothy Ruel},
  pdfkeywords={Directly standardized rate, Integrated likelihood ratio
statistic, Maximum integrated likelihood estimator, Profile
likelihood, Weighted sum, Zero score expectation parameter},
  colorlinks=true,
  linkcolor={blue},
  filecolor={Maroon},
  citecolor={Blue},
  urlcolor={Blue},
  pdfcreator={LaTeX via pandoc}}


\begin{document}


\def\spacingset#1{\renewcommand{\baselinestretch}%
{#1}\small\normalsize} \spacingset{1}


%%%%%%%%%%%%%%%%%%%%%%%%%%%%%%%%%%%%%%%%%%%%%%%%%%%%%%%%%%%%%%%%%%%%%%%%%%%%%%

\title{\bf Integrated Likelihood Inference in Poisson Distributions}
\author{
Timothy Ruel\\
Department of Statistics and Data Science, Northwestern University\\
}
\maketitle

\bigskip
\bigskip
\begin{abstract}
The text of your abstract. 200 or fewer words.
\end{abstract}

\noindent%
{\it Keywords:} Directly standardized rate, Integrated likelihood ratio
statistic, Maximum integrated likelihood estimator, Profile
likelihood, Weighted sum, Zero score expectation parameter
\vfill

\newpage
\spacingset{1.9} % DON'T change the spacing!

\section{Introduction}\label{sec-intro}

Suppose the random variables \(N_i\), \(i = 1, ..., m,\) each have
independent Poisson distributions such that \(\text{E}(N_i) = \theta_i\)
and
\(\theta = (\theta_1, ..., \theta_m) \in \Theta \subset \mathbb{R}^m_+\).
The purpose of this paper is to consider likelihood-based inference for
a real-valued parameter of interest \(\psi = g(\theta)\), where
\(g: \Theta \to \Psi\) is a known twice continuously differentiable
function.

The log-likelihood function of the model is given by

\begin{equation}\phantomsection\label{eq-1}{
\ell(\theta) = \sum_{i=1}^m \big[N_i \log(\theta_i) - \theta_i\big].
}\end{equation} Note that this is a function of the full
\(m\)-dimensional vector parameter \(\theta\), but the parameter of
interest \(\psi\) is just a scalar. This reduction in dimension induces
a nuisance parameter \(\lambda\) in the model that typically must be
eliminated from the log-likelihood function before inference regarding
\(\psi\) can be conducted. The standard procedure for doing so involves
choosing some method with which to summarize \(\ell(\theta)\) over its
possible values while holding \(\psi\) fixed in place. In the general
case where both \(\psi\) and \(\lambda\) are defined implicitly
(i.e.~\(\psi\) isn't just equal to one of the components of \(\theta\)),
this effectively reduces \(\ell(\theta)\) to a simple function of
\(\psi\) alone, having replaced each dimension of \(\theta\) that
depends on \(\lambda\) with a static summary of its range of values. We
call this new function a pseudo-log-likelihood function for \(\psi\) and
denote it as \(\ell(\psi)\).

Perhaps the most straightforward method of summarization we can use to
construct \(\ell(\psi)\) is to maximize \(\ell(\theta)\) over all
possible of values of \(\theta\) for a fixed value of \(\psi\). This
yields what is known as the \emph{profile} log-likelihood function,
formally defined as \begin{equation}\phantomsection\label{eq-2}{
\ell_p(\psi) = \sup_{\theta \in \Theta: \> g(\theta) = \psi} \ell(\theta).
}\end{equation}

Suppose we are interested in estimating the weighted sum of a group of
Poisson means corresponding to \(n\) independent populations, where
\(n\) is a known positive integer. Note that the maximum likelihood
estimate (MLE) for \(\theta_i\) is simply \(\hat{\theta}_i = x_i\), the
observed value of \(X_i\). Consider the weighted sum
\[Y = \sum_{i=1}^n w_iX_i,\] where each \(w_i\) is a known constant
greater than zero.

The purpose of this paper is to consider likelihood- and
pseudolikelihood-based inference for the real-valued parameter of
interest \[\psi \equiv \text{E}(Y) = \sum_{i=1}^n w_i\theta_i.\] In
particular, we will analyze the performance of point and inverval
estimates for \(\psi\) based on the integrated likelihood function and a
proposed modification to it. Similar estimates obtained from the profile
likelihood will be used as a benchmark.


  \bibliography{bibliography.bib}


\end{document}
